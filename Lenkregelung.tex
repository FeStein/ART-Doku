\chapter{Lenkregelung}\label{sec:Lenkregelung}
In diesem Abschnitt wird die Lenkregelung des Systems betrachtet (Aufgabe 2). Da
beide Fahrzeuge mit den gleichen Sensoren ausgestattet sind, ist die Regelung
für beide Fahrzeuge gleich und es wird lediglich ein Fahrzeug betrachtet. Ziel
war die Regelung mit einem PID-Regler, wobei der PID-Regler Block von Simulink
nicht verwendet werden durfte.
\begin{figure}[hbt]
\centering
\begin{subfigure}{\textwidth}
    \centering
    \includegraphics*[width=\textwidth]{figures/Lenkregelung.png}
\end{subfigure}
    \caption{Übersicht der Lenkregelung beider Fahrzeuge. Beinhaltet die
        Generierung der Stellgröße (1) sowie den PID-Regler (2).
    \label{fig:lenkregelung}}
\end{figure}    
In einem ersten Schritt haben wir uns die Eingangsgröße für den Regler überlegt.
Da zwei Farbsensoren vorhanden sind (jeweils rechts und links), müssen diese so
verrechnet werden, dass lediglich ein zusammengefasstes Signal vorhanden ist.

Der Sensor gibt durchgehend einen positiven Wert zurück, allerdings ist der Wert
geringer, je mehr schwarz erkannt wird. Somit verringert sich der Wert des
linken Sensors bei der Einfahrt in eine Linkskurve, der Wert des rechten Sensors
erhöht sich dabei da weniger bzw. analog mehr Licht reflektiert wird. Bei einer
perfekten Geräteausfahrt sind die Werte beider Sensoren gleich
($c_\mathrm{links} = c_\mathrm{rechts} = 50$). Um beide Werte zu verschmelzen,
werden die Werte einfach subtrahiert:
\begin{align*}
    c = c_\mathrm{links} - c_\mathrm{rechts}
\end{align*}
Für die Geräteausfahrt wird die Stellgröße $c$ dementsprechend $0$, da beide
Sensoren den gleiche Wert annehmen. Dies ist gleichzeitig auch das Regelziel für
den PID-Regler. Durch die Subtraktion der beiden Werte wird die Stellgröße stark
negativ für eine Linkskurve ($c_\mathrm{links}$ wird kleiner und
$c_\mathrm{rechts}$ wird größer). Bei einer Rechtskurve wäre dies genau
umgekehrt, jedoch kommt dies nicht in unserem Modell vor. In
Abbildung~\ref{fig:lenkregelung} ist dieser Teil mit (1) markiert. Der Target
Block wird dabei genutzt um die Regelabweichung zu berechnen. Benötigt wird er
eigentlich nicht, da das Regelziel $c=0$ ist.

Im zweiten Schritt wurde dann ein PID-Regler in Simulink implementiert. Die
jeweiligen Parameter $k_P$, $k_I$ und $k_D$ sind in den jeweiligen Gain Blöcken
in Simulink (Car\_2\_Steering\_P etc.). Durch ausprobieren sind wir auf die
folgenden idealen Werte für den PID-Regler gekommen:
\begin{align*}
    k_P = 1 \\
    k_I = 0 \\
    k_D = 0
\end{align*}

