\chapter{Dynamik}\label{chp:dynamik}
In diesem Teil geht es um die erste Aufgabe. Ziel ist die Bestimmung der Dynamik
der Fahrzeuge und darauf aufbauend die Zustandsraumdarstellung. Anschließend
sollen die relevanten Parameter identifiziert werden, wobei die Masse der
Fahrzeuge mit $m=1.465 \mathrm{kg}$ gegeben ist.
%-------------------------------------------------------------------------------
\begin{figure}[hbt]
\centering
\begin{subfigure}{0.15\textwidth}
    \centering
    \includegraphics*[width=\textwidth]{figures/FKB_Auto.png}
\end{subfigure}
    \caption{Freikörperbild eines Einspurmodells für das Fahrzeug.
    \label{fig:fkb_auto}}
\end{figure}    
%-------------------------------------------------------------------------------
Zunächst wurde ein Freikörperbild für das betrachtete Auto erstellt, analog zum
vereinfachten Einspurmodell
\footnote{\url{https://de.wikipedia.org/wiki/Einspurmodell}} eines Fahrzeugs.
Als Kräfte wirken lediglich die Antriebskraft $F_a$ die später als
Eingangsgröße betrachtet wird, sowie die Widerstandskraft $F_w$. In der
Widerstandskraft $F_w$ sind dabei alle Widerstände (Reibung, Trägheit, etc.)
zusammengefasst. Dadurch ergibt sich folgende Bewegungsgleichung für ein
einzelnes Auto im Zeitbereich:
\begin{align*}
    \sum F = m \frac{\diff^2 x}{\diff t^2} = F_A -F_W = -k_w \frac{\diff
    x}{\diff t} + k_a \cdot u(t)
\end{align*}
Die Konstanten $k_a$ und $k_w$ beinhalten dabei jeweils die Parameter für den
Antrieb, sowie die Dämpfung des Systems und müssen später genau bestimmt werden.
Für Frequenzbereich ergibt sich nun:
\begin{align*}
    X(s) \cdot (m \cdot s^2 + k_w \cdot s) = U(s) \cdot k_a \\
    G(s) = \frac{X(s)}{U(s)} = \frac{\frac{k_a}{k_w}}{\frac{m}{k_w} \cdot s^2 + s}
\end{align*}
Die Übertragungsfunktion ähnelt dabei einem PT1 Glied, wodurch die unbekannten
Parameter $k_a$ und $k_w$ durch das Tankentenverfahren bestimmt werden können.
Das Verfahren ist im matlab live Skript
\textit{Uebertragungsfunktion\_test.mlx} zu sehen. Schlussendlich haben wir für
die Parameter die folgenden Werte erhalten.
\begin{align*}
    k_w^{\mathrm{bestimmt}} = 7.0886\\
    k_a^{\mathrm{bestimmt}} = 0.0315 
\end{align*}
Danach wurden uns die folgenden idealen Werte vorgegeben, mit denen wir auch
weiter unsere Regelung gebildet haben:
\begin{align*}
    k_w = k_w^{\mathrm{ideal}} = 7.178\\
    k_a = k_a^{\mathrm{ideal}} = 1.603 
\end{align*}
Anschließend wurde die oben hergeleitete DGL, welche die Dynamik des Systems
beschreibt in eine Zustandsraumdarstellung gebracht:
\begin{align*}
\displaystyle\begin{bmatrix}
    x_1\\
    x_2
\end{bmatrix} &= 
\begin{bmatrix}
    x\\
    \frac{\diff x}{\diff t}
\end{bmatrix}
\\
\displaystyle\begin{bmatrix}
    \frac{\diff x_1}{\diff t}\\
    \frac{\diff x_2}{\diff t}
\end{bmatrix} &= 
\begin{bmatrix}
    x_2\\
    \frac{\diff^2 x}{\diff t^2}
\end{bmatrix}
\\
\displaystyle\begin{bmatrix}
    \frac{\diff x_1}{\diff t}\\
    \frac{\diff x_2}{\diff t}
\end{bmatrix} &= 
\begin{bmatrix}
    0 & 1 \\
    -\frac{k_w}{m} & 0
\end{bmatrix}
\begin{bmatrix}
    x_1\\
    x_2
\end{bmatrix} + 
\begin{bmatrix}
    0\\
    \frac{k_a}{m}
\end{bmatrix} u
\\
\displaystyle\begin{bmatrix}
    y_1 \\
    y_2
\end{bmatrix} &=
\begin{bmatrix}
    0 & r
\end{bmatrix}
\begin{bmatrix}
    x_1\\
    x_2
\end{bmatrix}
\end{align*}
