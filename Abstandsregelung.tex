\chapter{Abstandsregelung}\label{chp:Abstandsregelung}
\begin{figure}[hbt]
\centering
\begin{subfigure}{0.69\textwidth}
    \centering
    \includegraphics*[width=\textwidth]{figures/abstand_aufgabenstelllung.png}
\end{subfigure}
    \caption{Schematischer Aufbau der Regelung in der Aufgabenstellung
    \label{fig:abstand_as}}
\end{figure}    

%-------------------------------------------------------------------------------
\section{Übersicht Gesamtsystem}\label{sec:Übersicht Gesamtsystem}
ToDo: Gesamtsystem müsste auch mal fertig sein

%-------------------------------------------------------------------------------
\section{Zustandsgenerator}\label{sec:zustandsgenerator}
Ziel des Zustandsgenerators ist es die jeweiligen Positionen der Fahrzeuge aus
den gegebenen Sensoren (Position des rechten/linken Rads, Ultraschallsensor) zu
generieren. Eine Übersicht ist in Abbildung~\ref{fig:zustandsgenerator}
dargestellt.
\begin{figure}[hbt]
\centering
\begin{subfigure}{0.79\textwidth}
    \centering
    \includegraphics*[width=\textwidth]{figures/zustandsgenerator.png}
\end{subfigure}
    \caption{Übersicht über die Implementierung des Zustandsgenerators
    \label{fig:zustandsgenerator}}
\end{figure}    
Zunächst wird der kommulierte Winkel zwischen den beiden Rädern gemittelt. Dies
ist notwendig, da das rechte und das linke Rad sich unterschiedlich schnell
drehen bei einer Kurvenfahrt und somit auch eine unterschiedliche Position
angeben würden. Durch eine anschließende Multiplikation mit dem Reifenradius $r$
erhält man die Position des zweiten Fahrzeugs in Metern.

Für die Position des ersten Fahrzeugs wird auf die Position des ersten Fahrzeugs
der gemessene Abstand des Ultraschallsensors addiert. Somit erhält man die
Position des zweiten Fahrzeugs in Abhängigkeit des Ultraschall Sensors.
\begin{figure}[hbt]
\centering
\begin{subfigure}{0.49\textwidth}
    \centering
    \includegraphics*[width=\textwidth]{figures/zustand_position.png}
\end{subfigure}
    \caption{Plot der Position des 1. und 2. Fahrzeugs in Abhängigkeit der Zeit.
    \label{fig:zustand_position}}
\end{figure}    
In Abbildung~\ref{fig:zustand_position} ist die jeweilige Position in
Abhängigkeit der Zeit dargestellt. Die rosa Kurve (Fahrzeug 2) zeigt einen
stetigen Anstieg, nachdem das Fahrzeug die Fahrt aufgenommen hat. Die grüne
Kurve (Fahrzeug 1) zeigt einen um $0.5 \mathrm{m}$ verschobenen Verlauf, der
über den initialen Abstand zustande kommt. Allerdings weist der Verlauf immer
wieder Sprünge auf. Diese kommen durch die Kurvenfahrten Zustande, bei denen der
Ultraschallsensor den Kontakt zum Fahrzeug verliert. Die generierte Position des
Fahrzeugs 1 ist somit nicht vollständig und muss in diesen Zeiträumen korrigiert
werden. Dies geschieht mit dem in Abschnitt~\ref{sec:kalmanfilter} beschriebenen
Kalmanfilter. Basis dafür ist allerdings eine Fallunterscheidung, die
automatisch erkennt, ob sich das Fahrzeug gerade in einer Kurve befindet oder
nicht. Diese wird im Abschnitt~\ref{sec:fallunterscheidung} beschrieben.

%-------------------------------------------------------------------------------
\section{Fallunterscheidung}\label{sec:fallunterscheidung}

%-------------------------------------------------------------------------------
\section{Kalmanfilter}\label{sec:kalmanfilter}

%-------------------------------------------------------------------------------
\section{Lueneberger Beobachter}\label{sec:Lueneberger Beobachter}

%-------------------------------------------------------------------------------
\section{Zustandsrückführung}\label{sec:rueckfuehrung}

%-------------------------------------------------------------------------------
\section{Regelung}\label{sec:Regelung}

%-------------------------------------------------------------------------------
\subsection{PID-Regler}\label{subsec:PID-Regler}

%-------------------------------------------------------------------------------
\subsection{Optimale Regelung}\label{subsec:optimale_regelung}
